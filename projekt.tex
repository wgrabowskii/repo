\documentclass[12pt]{article}
\usepackage[T1]{fontenc}
\usepackage[utf8]{inputenc} 
\usepackage{amsmath}
\usepackage{amsfonts}
\usepackage{amssymb}
\usepackage[polish]{babel}
\usepackage{graphicx}
\usepackage{subfig}
\title{Projekt LaTeX}
\author{Wojciech Grabowski}
\date{12 stycznia 2016}

\begin{document}
\maketitle
\tableofcontents
\newpage

\section{Strona tekstu}

Kawałek streszczenia księgi I 'Pana Tadeusza'


Pan Tadeusz rozpoczyna się słynną inwokacją:



Litwo! Ojczyzno moja! ty jesteś jak zdrowie.
Ile cię trzeba cenić, ten tylko się dowie,
Kto cię stracił. Dziś piękność twą w całej ozdobie
Widzę i opisuję, bo tęsknię po tobie.



Narrator zwraca się do Litwy, a następnie do Matki Boskiej. Wspominając ojczyznę prosi Bożą Rodzicielkę o opiekę i cudowne „powrócenie na ojczyzny łono”. 
Tymczasem jednak przenosi się wspomnieniami do kraju dzieciństwa, charakteryzując specyfikę przyrody nadniemeńskiej. Następnie zaprezentowany zostaje typowo polski dworek szlachecki. 
Po tym opisowym wstępie rozpoczyna się akcja utworu.
Do dworku powraca po 10 latach nauki młody panicz – Tadeusz Soplica. Ze wzruszeniem przebiega wszystkie pomieszczenia, wspominając, jak wyglądały w czasach jego dzieciństwa. 
Gdy dociera do pokoju, w którym mieszkał, ze zdziwieniem spostrzega, że naleć on musi do jakiejś młodej panienki. 
Zastanawiając się, kto też może tu mieszkać, podchodzi do okna i spostrzega nieznane sobie dziewczę, które w rannej białej halce stoi pośród ogrodu. 
Nagle dostrzegłszy kogoś, kogo zapewne wypatrywała, szybko wbiegła do pokoju. 
Gdy dostrzegła młodzieńca niezmiernie przestraszona i zmieszana szybko umknęła, nie zważając na jego zmieszanie i przeprosiny.

Tymczasem zawiadomiony o pojawieniu się gościa Wojski – daleki krewny i przyjaciel domu, wyszedł go powitać. 
Opowiada Tadeuszowi, że trafił z datą przyjazdu idealnie, gdyż mają dużo gości, a stryj pragnąłby znaleźć mu żonę. 
Następnie obaj panowie wychodzą razem na spotkanie Sędziemu (stryjowi Tadeusza i gospodarzowi dworku) i jego gościom, którzy bawili w pobliskim lesie. 
Niedługo potem spotykają całe towarzystwo i po krótkich powitaniach wszyscy wracają do dworu. 
Pod nieobecność Sędziego i Wojskiego woźny Protazy nakazał podać kolację w nieodległym zamku, dawnym dziedzictwie rodziny Horeszków. 
O zamek ten Sędzia toczy spór z Hrabim, dalekim krewnym Horeszków.

Następnie wszyscy zasiadają do wieczerzy zgodnie z licznymi obyczajami litewskimi, m.in. przestrzegając kolejności zasiadania przy stole czy usługując damom. 
Sędzia, widząc, że zamyślony Tadeusz nie zabawia swoich sąsiadek, wygłasza nawet przemowę na temat grzeczności. 

\newpage

\section{Tabela }
\begin{center}
\begin{tabular}{|c||l|l|l|}
\hline Sezon & 1 & 2 & 3 \\ \hline \hline
12/13 & Messi & Ronaldo & Falcao \\
13/14 & Ronaldo & Messi & D. Costa \\
14/15 & Ronaldo & Messi & Neymar \\ \hline
\end{tabular}\\
\vspace{7mm}
Królowie strzelców Ligi BBVA
\end{center}

\section{Pierwsze równanie} 
Dla dowolnych
\begin{math}
a,p,q,\mathbb{C} \in \mathbb{R}, k \in \mathbb{N} + 
\backslash\{1\}$ i $\Delta = p^2 - 4q < 0
\end{math} mamy:
\begin{equation}
{\int\frac{1}{(x-a)^k}dx} = \frac{1}{1-k}(x-a)^{1-k}+ \mathbb{C}
\label{pierwsze}
\end{equation}

\section{Drugie równanie} 
Dla
\begin{math}
a > 0, a \neq 1$ i $ c \in \mathbb{R}
\end{math}
\begin{equation}
\lim_{x \to 0} \frac{\log_a(x+1)}{x}=\frac{1}{\ln a}
\label{drugie}
\end{equation}

\section{Równanie equationarray}
\begin{eqnarray*}
f(x) & = & \cos x \\
f'(x) & = & -\sin x \\
\end{eqnarray*}

Równanie \ref{pierwsze} jest dobre.
Równanie \ref{drugie} również jest ok.


\vspace{5mm}
Paulo Coelho powiedział:
\begin{quote}
 Parówki na zimno nie mają duszy!
\end{quote}


\newpage

\section{Rysunki}
\begin{figure}[ht]
\begin{center}
\includegraphics[width=5cm,height=5cm,keepaspectratio]{BPLlogo.png}
\caption{Logo BPL}
\label{BPL}

\includegraphics[width=5cm,height=5cm,keepaspectratio]{BBVAlogo.png}
\caption{Logo BBVA}
\label{BBVA}
\end{center}
\end{figure}

Logo \ref{BPL} to logo ligi angielskiej.
Logo \ref{BBVA} to logo ligi hiszpańskiej.
\section*{}
Pierwsza książka \cite{saga3} jest trzecią częścią sagi o Wiedźminie, natomiast druga książka \cite{saga5} to piąta, ostatnia cześć tej sagi.

\newpage


\end{document}